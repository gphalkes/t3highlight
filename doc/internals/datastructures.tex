\documentclass[a4paper,twocolumn]{article}
\usepackage{times}
\usepackage[margin=0.4in]{geometry}
%~ \usepackage{fullpage}
\usepackage{xspace}

\title{libt3highlight Internal Data-Structure Design}
\author{G.P. Halkes}
\date{}

% Disable underscore as an active character, because we will be typing it a
% lot, and never to indicate subscripts.
\catcode`\_=12\relax
\newcommand{\state}{{\tt state_t}\xspace}
\newcommand{\highlight}{{\tt highlight_t}\xspace}

\begin{document}
\maketitle

\section{Conventions}
In this document we use the following conventions. Names refering to items in a
language definition file are written using {\it italics}. Names of data
structures and their members are written using {\tt monospace}.

\section{State Storage}
The base internal data structure is a the \state. This
represents a list of highlighting patterns, corresponding to a
{\it highlight} (usually written as {\it \%highlight}) list in the
language definition file. A {\it highlight} sub-list, as created by a
{\it start} pattern, is represented by a separate \state.

The toplevel {\tt highlight} list is always at index 0 in the list of \state{}s
in the {\tt t3_highlight_t}. groups of patterns defined in the {\tt define}
section of the language definition are stored as separate \state{}s.

Each \state consists of a list of \highlight{}s. Each \highlight corresponds to
either the {\it regex} or {\it start} pattern in a {\it highlight} list entry.
The {\it end} pattern of a state is put at either the start or the end of the
list of {\it highlight}s in the sub-state, depending on whether it is defined
before or after the {\it highlight} list. Thus, the {\it start} pattern is part
of the enclosing state, while the {\it end} pattern is part of the state itself.

\subsection{Next State}
The {\tt next_state} member of the \highlight indicates which \state should be
used after matching the {\tt regex}. The value {\tt NO_CHANGE} (-1), indicates
that there should be no change in state after matching. Values greater than
{\tt NO_CHANGE} (i.e.\ 0 and greater) indicate a specific state to change to,
which is used for {\it start} patterns and {\it use} patterns. Finally, values
of {\tt EXIT_STATE} (-2) and smaller, indicate that this pattern is an end
pattern, and the parent state ({\tt EXIT_STATE}) or one of its ancestors
(values smaller than {\tt EXIT_STATE}) should be returned to.

It is important to note that the state numbers named here (i.e.\ the indices in
the {\tt t3_highlight_t}s \state list), are not the same as the states saved in
the {\tt t3_highlight_match_t}. See Section~\ref{sec:state-tree} for details on
the correspondence between them.

\footnotetext[1]{ The {\tt regex} member will be set to {\tt NULL} for dynamic
{\it end} patterns and for {\it use} patterns. These can be distinguished by
the setting of the {\tt next_state} member, which will be {\tt EXIT_STATE} or
less for dynamic {\it end} patterns.}

\subsection{Use Patterns}
A {\it use} pattern is signified by a {\tt NULL} {\tt regex} member in the
\highlight\footnotemark[1]. This allows referencing other patterns without
incurring a large memory penalty. Furthermore, it allows for cyclic
referencing, which can be useful for recursive embeddings such as counting
parenthesis.

\subsection{Dynamic End Patterns}
If the state uses a dynamic {\it end} pattern, the regular expression for the
pattern can not be compiled, until the match of the corresponding {\it start}
pattern. To signal this, the {\tt regex} member of the {\it end} pattern's \highlight is set
to {\tt NULL}\footnotemark[1]. The actual end pattern is created at run time
and stored in the {\tt state_mapping_t} (c.f.\ Section~\ref{sec:state-tree}).

The \highlight corresponding to {\it start} pattern will have its {\tt dynamic}
member set to a {\tt dynamic_highlight_t}, to hold the supplementary data
required. Specifically, it will hold the name to be extracted (c.f.\ {\it
extract}) and the {\it end} pattern.

\subsection{On-entry States}


\section{Matching and the Flattened State Tree\label{sec:state-tree}}

\subsection{Use Patterns}
\subsection{Dynamic End Patterns}
\subsection{On-entry States}

\end{document}
