\documentclass[a4paper]{article}
\usepackage{times}
\usepackage[left=0.45in,right=0.45in,top=0.7in,bottom=1in]{geometry}
\usepackage{graphicx}
\usepackage{multicol}
\usepackage{xspace}

\usepackage[scaled=0.95]{inconsolata}

\title{libt3highlight Internal Data-Structure Design}
\author{G.P. Halkes}
\date{}

% Disable underscore as an active character, because we will be typing it a
% lot, and never to indicate subscripts.
\catcode`\_=12\relax

\begin{document}
\maketitle

\begin{multicols}{2}
\section{Conventions}
In this document we use the following conventions. Names refering to items in a
language definition file are written using {\it italics}. Names of data
structures and their members are written using {\tt monospace}.

\section{State Storage}
The base internal data structure is a the {\tt state_t}. This
represents a list of highlighting patterns, corresponding to a
{\it highlight} (usually written as {\it \%highlight}) list in the
language definition file. A {\it highlight} sub-list, as created by a
{\it start} pattern, is represented by a separate {\tt state_t}.

The toplevel {\it highlight} list is always at index 0 in the array of
{\tt state_t}s in the {\tt t3_highlight_t}. groups of patterns defined in the
{\it define} section of the language definition are stored as separate
{\tt state_t}s.

Each {\tt state_t} consists of a array of {\tt pattern_t}s. Each {\tt pattern_t}
corresponds to either the {\it regex} or {\it start} pattern in a
{\it highlight} list entry. The {\it end} pattern of a state is put at either
the start or the end of the array of {\tt pattern}s in the sub-state,
depending on whether it is defined before or after the {\it highlight} list.
Thus, the {\it start} pattern is part of the enclosing state, while the
{\it end} pattern is part of the state itself.

\subsection{Next State}
The {\tt next_state} member of the {\tt pattern_t} indicates which
{\tt state_t} should be used after matching the {\tt regex}. The value
{\tt NO_CHANGE} (-1), indicates that there should be no change in state after
matching. Values greater than {\tt NO_CHANGE} (0) indicate a specific state to
change to, which is used for {\it start} patterns and {\it use} patterns.
Finally, values of {\tt EXIT_STATE} (-2) and smaller, indicate that this
pattern is an end pattern, and the parent state ({\tt EXIT_STATE}) or one of
its ancestors (values smaller than {\tt EXIT_STATE}) should be returned to.

It is important to note that the state numbers named here (i.e.\ the indices in
the {\tt t3_highlight_t}s {\tt state_t} array), are not the same as the states
saved in the {\tt t3_highlight_match_t}. See Section~\ref{sec:state-tree} for
details on the correspondence between them.

\footnotetext[1]{The {\tt regex} member will be set to {\tt NULL} for dynamic
{\it end} patterns and for {\it use} patterns. These can be distinguished by
the setting of the {\tt next_state} member, which will be {\tt EXIT_STATE} or
less for dynamic {\it end} patterns.}

\subsection{Use Patterns}
A {\it use} pattern is saved as a separate {\tt state_t}, signified by a
{\tt NULL} {\tt regex} member in the {\tt pattern_t}\footnotemark[1]. This
allows referencing other patterns without incurring a large memory penalty.
Furthermore, it allows for cyclic referencing, which can be useful for
recursive embeddings such as shell varia ble references within variable
references.

\subsection{Dynamic End Patterns}
If the state uses a dynamic {\it end} pattern, the regular expression for the
pattern can not be compiled, until the match of the corresponding {\it start}
pattern. To signal this, the {\tt regex} member of the {\it end}'s
{\tt pattern_t} is set to {\tt NULL}\footnotemark[1]. The actual end pattern is
created at run time and stored in the {\tt state_mapping_t}
(c.f.\ Section~\ref{sec:state-tree}).

\footnotetext[2]{The {\tt extra} member will be set for both dynamic end
patterns and for on-entry states.}

The {\tt pattern_t} corresponding to {\it start} pattern will have its
{\tt extra} member set to a {\tt pattern_extra_t}, to hold the supplementary
data required\footnotemark[2]. Specifically, it will hold the name to be
extracted (c.f.\ {\it extract}) and the {\it end} pattern.

\subsection{On-entry States}
On-entry states are saved as a array in the {\tt extra} member\footnotemark[2].
For each on-entry state, the index in the {\tt state_t}s array is saved.
Furthermore, because on-entry states can use dynamic end patterns referreing to
the enclosing normal state, each on-entry state optionally holds the end
pattern required for matching the end state.

\section{Matching and the Dynamic State Tree\label{sec:state-tree}}
The main design goal of libt3highlight is to have a single integer state at the
start of each line. Originally, this was achieved by having a simple state
tree, stored as an array, where each node had a parent index. However, with the
introduction of dynamic end patterns, this could not work anymore due to the
extra dynamic state that needs to be stored. Thus instead of using the
{\tt state_t} array index as the start-of-line state, an intermediate data
structure was introduced. This is the flattened state tree (DST), which is
stored in the {\tt t3_highlight_match_t}.

The DST is a vector of {\tt state_mapping_t} objects. Each
{\tt state_mapping_t} holds an index in the {\tt state_t}s array in the
{\tt t3_highlight_t} structure ({\tt highligh_state}), a {\tt parent} member
referring to the previous DST state visited in matching, and an optional
{\tt extra} member holding data about dynamic end patterns.

The {\tt parent} member thus refers to the {\tt state_mapping_t} object that
represents the previous DST state, and transitively to the whole path taken in
the {\tt state_t} tree. This is important when trying to find an existing entry
in the {\tt state_mapping_t} array which corresponds to the next state, as it
obviates the need to check the entire chain of parent indices. Three conditions
must be met for the entry to be matched:

\begin{itemize}
\item The {\tt parent} member must equal the current {\tt state_mapping_t}.
\item The {\tt highlight_state} member must equal the {\tt next_state} member
	in the {\tt pattern_t}.
\item The {\tt extra} member of {\tt pattern_t} must be unset or have its
	{\tt dynamic_name} unset, or the {\tt extracted} member in the
	{\tt dynamic_state_t} referred to by the {\tt state_mapping_t} must be
	equal to the text extracted from the match.
\end{itemize}

When these three conditions are met, the existing DST-state can be reused. If
no DST state exists which matches the new state, a new state will be created.
It may seem that this will create a large and ever-growing number of DST states.
However, note that for the case where the highlighting patterns form a simple
tree without dynamic patterns, the maximum number of states is equal to the
number of {\tt state_t}s in the {\tt t3_highlight_t}. Only for circular {\it use}
patterns and dynamic end patterns can a potentially infinite number of states
be created. But, even though this is possible, it is very unlikely that the
number of states grows beyond a reasonable number in practice.

\begin{description}
\item[Use patterns] are never used in the flattened state tree. When a
{\tt state_t} corresponding to a {\it use} pattern is encountered during matching, a
recursion step is made, which checks all the patterns in the {\it use} pattern
and returns the results.
\item[Dynamic-end patterns] require that a new regex is compiled based on the
matched text and the saved regex pattern. This is a fairly expensive operation.
\item[On-entry states] are simply added to the DST like any other state, when
a state is entered which specifies them.
\end{description}
\end{multicols}

\begin{figure}
\centering
\includegraphics[width=.9\textwidth]{structs}
\caption{An overview of the data structures used in libt3highlight.}
\end{figure}

\end{document}
